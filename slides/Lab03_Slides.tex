% Lab 03: Text Chunking
\documentclass[aspectratio=169]{beamer}
\usetheme{Madrid}
\usecolortheme{whale}
\setbeamertemplate{navigation symbols}{}
\setbeamertemplate{footline}[frame number]

\usepackage{listings}
\usepackage{tikz}

\lstset{
    language=Python,
    basicstyle=\ttfamily\small,
    keywordstyle=\color{blue}\bfseries,
    stringstyle=\color{red},
    frame=single,
    backgroundcolor=\color{gray!10},
    breaklines=true
}

\title{Lab 03: Text Chunking}
\subtitle{Splitting Documents for RAG Systems}
\author{CSI403 - Full Stack Program Development}
\date{}

\begin{document}

\begin{frame}
    \titlepage
\end{frame}

\begin{frame}{Why Chunking?}
    \begin{center}
        \textbf{LLMs have token limits!}
    \end{center}
    
    \begin{itemize}
        \item GPT-4: 8K-128K tokens
        \item Claude: 100K-200K tokens
        \item Local LLMs: 2K-8K tokens
    \end{itemize}
    
    \vspace{0.5cm}
    
    \begin{block}{Solution: Split into Chunks}
        Large Document → [Chunk 1] [Chunk 2] [Chunk 3] ...
    \end{block}
\end{frame}

\begin{frame}[fragile]{Simple Chunking: String Slicing}
    \begin{lstlisting}
text = "ABCDEFGHIJKLMNOPQRSTUVWXYZ"

# Slice from index 0 to 10
chunk1 = text[0:10]   # "ABCDEFGHIJ"
chunk2 = text[10:20]  # "KLMNOPQRST"
chunk3 = text[20:26]  # "UVWXYZ"
    \end{lstlisting}
    
    \begin{block}{Syntax: text[start:end]}
        Returns characters from index start to end-1
    \end{block}
\end{frame}

\begin{frame}[fragile]{Automatic Chunking with Loop}
    \begin{lstlisting}
def simple_chunk(text, chunk_size):
    chunks = []
    for i in range(0, len(text), chunk_size):
        chunk = text[i:i+chunk_size]
        chunks.append(chunk)
    return chunks

result = simple_chunk("ABCDEFGHIJ", 3)
# ['ABC', 'DEF', 'GHI', 'J']
    \end{lstlisting}
\end{frame}

\begin{frame}[fragile]{Chunking with Overlap}
    \textbf{Overlap preserves context between chunks!}
    
    \begin{lstlisting}
def chunk_with_overlap(text, size, overlap):
    chunks = []
    start = 0
    while start < len(text):
        end = start + size
        chunks.append(text[start:end])
        start = end - overlap
    return chunks

# size=10, overlap=3
# [ABCDEFGHIJ] [HIJKLMNOPQ] [OPQRSTUVWX]
    \end{lstlisting}
\end{frame}

\begin{frame}{Summary}
    \begin{itemize}
        \item \textbf{Chunking} = splitting text into smaller pieces
        \item Use \texttt{text[start:end]} for slicing
        \item Use \texttt{range(0, len(text), chunk\_size)} for looping
        \item \textbf{Overlap} helps maintain context
    \end{itemize}
    
    \begin{block}{Next Lab}
        Lab 04: Text Search - finding relevant chunks!
    \end{block}
\end{frame}

\begin{frame}
    \begin{center}
        \Huge \textbf{Questions?}
        
        \vspace{1cm}
        
        \Large Let's start the Tutorial!
    \end{center}
\end{frame}

\end{document}
